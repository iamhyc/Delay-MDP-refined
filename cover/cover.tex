\documentclass[10pt,draftclsnofoot,onecolumn]{article}
\usepackage{a4wide}

\renewcommand{\baselinestretch}{1.5}

\usepackage{amsthm}
\usepackage{amsfonts,amssymb}
\usepackage{amsmath} % Required for some math elements
{
      \newtheorem{definition}{Definition}
      \newtheorem{assumption}{Assumption}
      \newtheorem{problem}{Problem}
      \newtheorem{lemma}{Lemma}
      \newtheorem{theorem}{Theorem}
      \newtheorem{corollary}{Corollary}
      \newtheorem{claim}{Claim}
      \newtheorem{remark}{Remark}
      \newtheorem{policy}{Policy}
}

\usepackage[linesnumbered,vlined,ruled]{algorithm2e}
\usepackage{graphicx}
\usepackage{subfigure}

\usepackage{cite}
%\usepackage{hyperref}
%\usepackage[numbers,sort&compress]{natbib}

\usepackage{xcolor}
%\usepackage[usenames,dvipsnames]{color}
%\usepackage{mathrsfs}
\usepackage{pifont}% http://ctan.org/pkg/pifont
\newcommand{\cmark}{\ding{51}}%
\newcommand{\xmark}{\ding{55}}%
\usepackage{rotating}

\usepackage{amsfonts,amssymb}

%\usepackage{ulem}
%\usepackage{cancel}

\usepackage{diagbox}
\usepackage{sgame}
%\usepackage{ulem}
%\usepackage{sgamevar}
%\usepackage{egameps}
%\usepackage{caption2}

\usepackage{caption}
\usepackage{sidecap}
\usepackage{url}

%\usepackage{tablefootnote}
\usepackage{booktabs}
\usepackage{threeparttable}

\usepackage{array}

\newcommand{\fixit}[1]{{\leavevmode\color{red}#1}}

\begin{document}

Dear Editor and Reviewers,

We are pleased to submit our paper titled ``Distributed Job Dispatching in Edge Computing Networks with Random Transmission Latency: A Low-Complexity POMDP Approach" for possible publication in IEEE Internet of Things Journal.
% IEEE Transactions on Mobile Computing.

A preliminary version with part of the results of this submission appeared in IEEE MSN 2020 held in Tokyo, Japan.
Compared with the conference  publication, the current submission makes the following novel contributions:
\begin{enumerate}
% \item In the INFOCOM paper, we only solve the HetNets uplink scheduling problem using our  proposed algorithms. However, our solution framework can be applied to  a series of scheduling problems in multi-tier networks. In this submission, we add a section that extensively discusses how to extend our solution to other important multi-tier networks including Cloud-RAN and Mobile Edge Computing. [Please refer to Section 5]
\item In this manuscript, \fixit{we add a section of ``Background and Motivation'' to justify the setting of the limited backhaul capacity, and the motivation why we adopt reinforcement learning due to the diversity and time-varying characteristics in HetNets. [Please refer to Section 2]}
\item In the MSN paper, \fixit{we adopt discounted accumulated cost over infinite time-horizon when modelling the HetNets uplink scheduling problem. In this manuscript, we model the problem with a more realistic cost function, i.e., the average cost over infinite time-horizon, which is naturally derived from the real scenario [Please refer to Problem 1]. Our proposed algorithm has two approaches for the average cost model and the discounted cost model, respectively. For the average cost model, we add a new Bellman equation [Eqn. (9)], new reference states [Eqn. (14)], and a new $\mathbb Q$-function updating rule [Eqn. (18)]. Moreover, we bridge the gap between the average cost model and the discounted cost model by providing a justification of the benefits of discounted cost model. [Please refer to Section 4.1]}
\item In the MSN paper, \fixit{we assume the joint detection is a binary decision, i.e., each frame's signal can only be either jointly detected or local detected. In this submission, we improve this decision by allowing a portion of sub-frames to be locally detected. We further consider the case that the pico-BS may have limited detection ability that can only detect a limited number of sub-frames. [Please refer to Section 3.2]}
% add an evaluation here?
\item In the MSN paper, \fixit{we only give the end-to-end delay as the cost function. Actually, the cost function can be general to  cover other forms of cost. In this submission, we propose two other forms of cost functions: the throughput and the power consumption. We also explain why using the end-to-end delay as cost function already maximizes the throughput. [Please refer to Remark 1 in Section 3.4]}
\item In the MSN paper, \fixit{we did not elaborate the related works in details. In this submission, we extensively introduce the related works, and compare them with our work. [Please refer to Section 2]}
\item In this submission, \fixit{we add a new sensitivity experiment showing the percentage of local detection used with higher arrival rate. [Please refer to Fig. 6]}
\item In the MSN paper, \fixit{the proofs of NP-hardness, the data rate derivation, the convergence and the error\newcommand{\fixit}[1]{{\leavevmode\color{red}#1}} bound are omitted due to limited space. We complete all the proofs in the appendices in this submitted manuscript.  [Please refer to the Appendix A, B, C and D]}
%\item \tant{In the INFOCOM paper, we only investigated the HetNets uplink scheduling problem. Actually, our solution framework can be applied to  a series of scheduling problems in multi-tier networks. In this submission, we add a section that extensively discusses how to extend our solution to other important multi-tier networks including Cloud-RAN and Mobile Edge Computing. [Please refer to Section 5]}
\end{enumerate}

Based on the aforementioned extensions, we believe this submitted manuscript comprises a significant technical contribution compared with the previous conference version.\\

Thank you very much and we look forward to your feedbacks on this submission.\\

Best Regards,

Yuncong Hong on behalf of the authors
\vspace{+20mm}

\noindent [1] Y. Hong, B. Lv, R. Wang, H. Tan, Z. Han, H. Zhou, F. C. M. Lau, ``Online Distributed Job Dispatching with Outdated and Partially-Observable Information'', in Proc. \emph{IEEE MSN}, 2020\\
% \noindent [1] Z. Han, H. Tan, R. Wang, S. Tang and F. C. M. Lau, ``Online Learning based Uplink Scheduling in HetNets with Limited Backhaul Capacity'', in Proc. \emph{IEEE INFOCOM}, 2018\\
\end{document}
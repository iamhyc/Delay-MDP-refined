\IEEEtitleabstractindextext{
    \begin{abstract}
        random latency in edge computing network is just another big issue when considering distributed scheduler design in a large network, such as Metropolitan Area Network (MAN).

        % In this paper, we investigate online distributed job dispatching in an edge computing system residing in a Metropolitan Area Network (MAN), where multiple access points (APs) collect jobs from the mobile users and upload each job to one edge server for computation.
        % A signaling mechanism with periodic broadcast is introduced to facilitate cooperations among APs.
        % The signaling and job dispatching latency is random and non-negligible in MAN, and the APs have to update their job dispatching strategies according to partially received and outdated broadcast information.
        % Moreover, the fully-observed system state is discouraged as reception of all broadcast is time consuming.
        Therefore, we formulate the distributed optimization of job dispatching strategies at all the APs as a partially observable Markov decision process (POMDP), whose minimization objective is a discounted measurement of job delivery and computation time.
        % The conventional solution for POMDP is impractical due to huge time complexity.
        % In this paper, we propose a novel low-complexity solution framework for distributed job dispatching.
        Based on it, the optimization of job dispatching policy can be decoupled via an \emph{alternative policy iteration algorithm}, called \algname, so that the distributed policy iteration of each AP can be made according to partial and outdated observation.
        % An analytical performance lower bound is provided for the approximate MDP solution.
        Furthermore, we extend \algname~to handle a more general scenario where the statistics of job arrivals, uploading latency and job processing time are unknown.
        The extended \algname~leverages a novel and efficient reinforcement learning approach to online continuously improve the system performance.
        % Finally, we conduct extensive simulations based on the Google Cluster trace.
        % The evaluation results show that our policy can achieve as high as $20.67\%$ reduction in average job response time compared with heuristic baselines, and our algorithm consistently performs well under various parameter settings.
    \end{abstract}

    % Note that keywords are not normally used for peer-review papers.
    \begin{IEEEkeywords}
        Edge computing, metropolitan area network, partially-observable MDP, reinforcement learning.
    \end{IEEEkeywords}
}

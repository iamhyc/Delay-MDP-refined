% \IEEEtitleabstractindextext{
    \begin{abstract}    
        %NOTE: (1) background: distributed job dispatching with latency;
        Job dispatching is a fundamental problem in edge computing for load balancing among multiple edge servers. When implementing an edge computing system with distributed job dispatchers in an extensive network, such as a Metropolitan Area Network (MAN), the highly dynamic transmission latency is a non-negligible issue, which leads to outdated information sharing. Moreover, the fully-observed system state is discouraged as reception of all broadcast is time consuming. %The exchange of system status information would inevitably become outdated, and the reception of whole system information is discouraged.
        % the transmission latency is a non-negligible issue, such that the exchange of system status information would inevitably become outdated.
        % Moreover, as the latency might be highly dynamic in a MAN, the reception of whole information exchange is time-consuming and thus discouraged.
        % \hongyc{Most of the existing works did not take the variable transmission latency into consideration.}
        %NOTE: (3) problem formulation: online/distributed/job dispatching; the signaling mechanism (information sharing); claim partial&outdated again;
        In this paper, we investigate an online distributed job dispatching problem in edge computing, where multiple access points (APs) collect jobs and then dispatch each job to an edge server.
        The distributed dispatcher on each AP would receive partially and outdated information exchanged via periodic broadcast.
        % A signaling mechanism is introduced to facilitate periodic information broadcast among distributed dispatchers on each AP.
        %Due to the highly dynamic latency, the job dispatchers have to make decisions individually according to partially received and outdated broadcast information.
        %NOTE: (4) technical contribution: POMDP; low-complexity solution with bound;
        Hence, we formulate the distributed job dispatching problem by leveraging partially observable Markov decision process (POMDP) and propose a novel approximate MDP solution framework, called \algname, to avoid the huge time complexity of conventional POMDP solutions.
        Both analytical and semi-analytical performance lower bounds are derived for the approximate MDP solution. %for quick performance evaluation.
        Furthermore, we extend \algname~to handle a more general scenario where the priori knowledge of the system is absent.
        %NOTE: (5) solution and results: \tann{In this work, our solution and results.}
        % The evaluation results show that our policy can achieve as high as $20.67\%$ reduction in average job response time compared with heuristic baselines,
        Finally, extensive simulations based on the Google Cluster trace show that our policy can achieve the best performance compared with heuristic baselines, e.g., achieving $20.67\%$ reduction in average job response time, and consistently performs well under various parameter settings.
    \end{abstract}

    % Note that keywords are not normally used for peer-review papers.
    \begin{IEEEkeywords}
        Edge computing, metropolitan area network, partially-observable MDP.
    \end{IEEEkeywords}
% }

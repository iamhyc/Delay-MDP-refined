% \IEEEtitleabstractindextext{
    \begin{abstract}
        %NOTE: (1) random latency matters; why random latency matters;
        Random transmission latency is a non-negligible issue when implementing distributed schedulers of a edge computing system in a large network such as Metropolitan Area Network (MAN).
        With the random latency of data transmission among schedulers, the exchange of system status information would inevitably become outdated and degrade the system performance.
        %NOTE: (2) online/distributed/cooperative/job dispatching/outdated information
        In this paper, we investigate an online distributed job dispatching problem in such an edge computing system, where multiple access points (APs) collect jobs from the mobile Internet of Things (IoT) devices and offload each job to one edge server for computation.
        %NOTE: (3) the signaling mechanism (information sharing); partial information;
        A signaling mechanism is introduced to facilitate periodic information broadcast among distributed schedulers on each AP.
        % As the signaling latency is random and reception of whole broadcast information in a MAN is time-consuming
        The schedulers have to make job dispatching decisions individually according to partially received and outdated broadcast information.
        %NOTE: (4) technical contribution: POMDP; low-complexity solution with bound; RL;
        Hence, we formulate the distributed job dispatching problem of the overall edge computing system by leveraging the partially observable Markov decision process (POMDP) and propose a novel approximate MDP solution framework, called \algname, to avoid the huge time complexity of conventional POMDP solutions.
        Both analytical and semi-analytical performance lower bounds are derived for the approximate MDP solution for quick performance evaluation.
        Furthermore, we extend \algname~to handle a more general scenario where the statistics of job arrival and transmission latency are unknown.
        %NOTE: (5) the simulation and performance improvement
        % Finally, we conduct extensive simulations based on the Google Cluster trace.
        % The evaluation results show that our policy can achieve as high as $11.21\%$ reduction in average job response time compared with heuristic baselines,
        Finally, the evaluation based on the Google Cluster trace shows that our policy can achieve the best performance compared with heuristic baselines, and consistently performs well under various parameter settings.
        % and our algorithm consistently performs well under various parameter settings.
    \end{abstract}

    % Note that keywords are not normally used for peer-review papers.
    \begin{IEEEkeywords}
        Edge computing, metropolitan area network, partially-observable MDP, reinforcement learning.
    \end{IEEEkeywords}
% }

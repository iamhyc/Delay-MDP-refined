
\IEEEraisesectionheading{
    \section{Introduction}
    \label{sec:introduction}
}
%NOTE: (1) MEC Introduction: Background, Challenge and Motivation
%NOTE: (1a) general background and bluffing
\IEEEPARstart{E}{dge} computing is gaining more and more attentions with the prosperity of computation-intensive and delay-sensitive mobile applications. % computation-intensive, energy-hungry
The edge servers are deployed in closer proximity to mobile users than traditional cloud computing infrastructure, which alleviate the communication overhead and enable computation offloading from mobile devices.
%NOTE: (1b) why multiple edge servers deployed
On the other side, the edge servers are usually deployed with limited and non-scalable computation resource compared to cloud servers.
Hence, the deployment of many standalone edge computing servers in the network is favored.
In a general edge-cloud system architecture \cite{MEC-SURVEY}, the offloaded jobs from mobile devices could be delivered to one of the edge servers considering the transmission latency, job processing time, back pressure and etc.
%NOTE: (1c) introduce the access points
The entity which decides edge servers for offloaded jobs is called \emph{access points} (APs) throughout this paper.
Specifically, the APs function as gateways to collect offloaded jobs from mobile devices in its service area and make dispatching decision for each job.

%NOTE: (2) Motivation with MAN and Distributed Dispatchers
%NOTE: (2a) why cooperation is needed; cooperation in MAN is another challenge
To achieve better overall performance (e.g., average job response time) of the edge computing system, the cooperation of jobs dispatching among distributed APs is one of the major challenges.
Most of the existing works simply assume a centralized job dispatcher design, where the dispatcher timely holds the global system status and distributes the dispatching decisions to all the APs without delay.
However, the real-world practice prefers edge computing system deployment at Metropolitan Area Network (MAN) scale \needref{MAN-scale applications}, i.e., the latency of information sharing among APs and edge servers is non-negligible, which discourages the centralized dispatcher design.
According to the MAN performance analysis in \cite{MAN-LATENCY}, the end-to-end transmission latency varies a lot with respect to different hours of day and devices' locations in a MAN, which implies the randomness of job uploading latency from APs to edge servers and signaling latency (i.e., the transmission latency of the shared {system status} information).
%NOTE: (2b) challenge brought by random latency
There has been a number of existing literature considering random transmission latency of job delivery in the edge computing network \cite{latency-EDGE19,MOBIHOC19-ZhouZ,IOTJ18-FanQ,TOC19-LiuC,JSAC19-AlameddineHA}.
However, there are only a few works considering random signaling latency of information sharing among distributed dispatchers \cite{tan-online,TWC18-LyuX}.
In fact, it is full of challenge to consider random transmission latency in both job uploading and signaling.
Firstly, the centralized dispatcher design is discouraged for outdated system information and unpredictable signaling latency.
Secondly, the cooperation of distributed dispatchers suffers from significant signaling overhead, and the random job uploading latency causes the inconsistency of system information at different dispatchers because the arrivals on edge servers are unknown in advance.
To conclude, the random transmission latency may introduce ineliminable estimation error on the number of jobs in the system.
% {i.e., the cooperation of APs are assumed inside one entity like a operator, but across a large network which is larger than a LAN.}
% {In addition, each AP also suffers from signaling latency, which is the time consumed for each AP to collect system state information under some signaling mechanism.}


% NOTE: (3) Our contributions
In this paper, we address the above challenges by leveraging a partially observable Markov decision process (POMDP) problem formulation, and a novel low-complexity approximate MDP solution framework is proposed.
Specifically, we consider a practical scenario where the APs cooperatively upload different types of jobs to different edge servers with random uploading latency and the APs receive the partial system state information with random signaling latency (i.e., the global system state is always partially available and outdated at the APs).
The major contributions in this paper are summarized below.
\begin{itemize}
    \item To optimize the distributed cooperative job dispatching design with outdated and partial information, a POMDP problem is formulated.
    Different from the conventional value or policy iteration of the Bellman's equations where global or historical system states are requested in numerical calculation, a novel low-complexity approximate MDP solution framework via \emph{alternative policy iteration} is proposed, where the dispatching policies of all APs are updated distributedly and alternatively on the {closed-form expression} of the approximate local value function.
    Thus, the complicated POMDP solution is avoided.
    % {where each AP collects the information only from the APs and edge servers in close proximity and make dispatching decisions with random signaling latency.}
    % {we directly derive the expression of approximate value function}
    \item A tight analytical cost lower bound is provided for the proposed distributed dispatching policy in the solution framework \algname. In the conventional approximate MDP methods, the performance is usually evaluated via numerical method as it is hard to obtain analytical performance bound.
    \item We extend our solution framework \algname~with a novel and efficient reinforcement learning approach to evaluate the approximate value function when the priori knowledge of the system randomness is absent.
    \item We conduct extensive simulations based on the Google Cluster trace, compared with three heuristic benchmarks. The evaluation results show that \algname~can achieve $20.67\%$ reduction in average job response time, and consistently perform well under various parameter settings of signaling latency, job arrival intensity and job processing time. {Moreover, the reinforcement learning algorithm converges fast.}
    % The extended \algname~could converge to the real parameter settings fast with negligible error
\end{itemize}

The remainder of this paper is organized as follows.
In Section \ref{sec:review}, we introduce some related works about job scheduling in edge computing systems.
In Section \ref{sec:model}, we elaborate the system model and the signaling mechanism with random transmission latency.
In Section \ref{sec:formulation}, we formulate the global-wise optimization of dispatching decisions at all APs as a POMDP.
In Section \ref{sec:algorithm}, we propose a novel low-complexity approximate MDP solution framework, called \algname.
% where the policy iteration could be applied distributedly on each AP {with partial and outdated information}.
In Section \ref{sec:rl-alg}, the solution framework is extended with reinforcement learning technique to online predict the unknown statistics.
The numerical analysis of the proposed solution is provided in Section \ref{sec:evaluation}, and the conclusion is drawn in Section \ref{sec:conclusion}.

We use the following notations throughout this paper: 
non-bold letters (e.g., $a, A$) are used to denote scalar values,
bold lowercase letters (e.g., $\mathbf{a}$) are used to denote column vectors,
bold uppercase letters (e.g., $\mathbf{A}$) are used to denote matrices,
and calligraphic letters (e.g., $\mathcal{A}$) are used to denote sets.
Using these notations, $[\mathbf{A}]_{i,j}$ and $\mathbf{A}'$ denotes the $(i,j)$-th element and transpose of matrix $\mathbf{A}$, respectively.
% $\mathbf{I}$ denotes the identity matrix.

\section{Related Works}
\label{sec:review}
%NOTE: (1) resource placement (cache-like problem), service migration
There have been a number of works aiming at reducing job response time by resource allocation and service migration in the edge computing system.
For example, in \cite{TON19-WangSq}, the edge servers are one-to-one bound to the base stations (BSs), and the job migration could be applied according to users' mobility traces via the backhaul network connecting the BSs.
However, according to a recent research \cite{INFOCOM19-WuC}, the resource re-allocation for running jobs on servers is hard to implement in practice, as it is hard for jobs migration among heterogeneous edge servers with different resource configurations.
Hence, it might be more important to optimize the job dispatching strategy at their arrival time.

%NOTE: (2) single-agent job dispatching, single UE/server
There also have been a number of works considering the job dispatching with a centralized dispatcher design which holds timely and complete knowledge of the edge computing system.
For example, in order to minimize the average job response time in the worst case, the authors in \cite{tan-online} designed an online algorithm for job dispatching in edge computing systems with fixed uploading latency.
In the scenario that BSs and edge servers are connected via software defined network (SDN), the authors in \cite{IOTJ18-FanQ} proposed a heuristic algorithm to dispatch the jobs to the closest edge servers according to geographical locations.
Considering random jobs arrival and job offloading to a single edge server, the authors in \cite{mdp-globecom,mdp-tvt} formulate the offloading problem as an infinite-horizon Markov decision process (MDP).
When the jobs can be dispatched to either edge servers or cloud servers with fixed uploading latency, the authors in \cite{MASS18-MengZ} formulated job dispatching problem as an integer linear programming to minimize the total uploading latency.
In the above works, a centralized dispatcher with complete and instant knowledge of the system status was assumed in the edge computing systems, which might be impractical.

%NOTE: (3) multiple-agent job dispatching
Hence, there are also some works considering the distributed job dispatching in edge computing systems.
For example, in order to minimize a weighted sum of total energy consumption and uploading latency, the authors in \cite{ToN-Xuchen2016} proposed a distributed job dispatching algorithm based on game theory to achieve the Nash equilibrium. 
Considering job migration at edge servers, the authors in \cite{ToN-xujie2018} optimized the edge computing performance in a distributed manner with limited energy resources via a congestion game framework.
In the scenario that APs cooperatively dispatch jobs with multiple edge servers, the authors in \cite{mdp-jcin} proposed a novel approximate MDP solution framework to alleviate the algorithm complexity and minimize the average job response time.
However, in the above works, the latency of information sharing among APs and edge servers is ignored.
In fact, due to the complicated network traffic, this latency might be significant, and the staleness of system state information at the dispatcher of a edge computing systems should be considered.

%NOTE: (4) stale-information based multi-agent related works
%FIXME: (Optional) if have time, rewrite this part
The staleness of information sharing among APs and edge servers may degrade the performance of the job dispatching algorithm in edge computing systems.
To the best of our knowledge, there are very limited works investigating this issue.
For example, the authors in \cite{JSAC17-LyuX} proposed a randomized policy via Lyapunov optimization approach to stabilize the queues in a MEC system with multiple IoT devices offloading jobs to one edge server, where \brlatency~is considered. 
In \cite{TWC18-LyuX}, the above approach was applied to the scenario that mobile devices offload jobs to each other via D2D link.
In the above two works, there is one centralized dispatcher in the system and the objective is to stabilize the transmission queues.
Hence, the existence of \brlatency~may not raise significant challenge to the algorithm design with Lyapunov optimization.
However, the design of distributed dispatchers with \brlatency~could be more challenging.
For example, the signaling latencies at distributed dispatchers could be different, and the synchronization of their dispatching decisions become infeasible.
Furthermore, taking the signaling overhead and the possibility of packet drop in the consideration, it is of more practical significance favor for the distributed dispatchers to make scheduling decisions based on locally observable system state information, instead of global system state information.
To our best knowledge, there is no appropriate optimization framework for the distributed dispatcher design with both \brlatency~and partially observable system state information to date.

%----------------------------------------------------------------------------------------%

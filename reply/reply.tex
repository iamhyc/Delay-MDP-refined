\documentclass[12pt, onecolumn]{IEEEtran}

\usepackage{cite}
% \usepackage[noend]{algpseudocode}
\usepackage{graphicx}
\graphicspath{ {./images/} }
\usepackage{amsmath,amsthm,amssymb,amsfonts}
\usepackage{mathtools}
\usepackage[dvipsnames]{xcolor}
\usepackage{dcolumn}
\usepackage[utf8]{inputenc}
\usepackage{soul}
\usepackage{array}
\usepackage{tabulary}
%---------------------------------------------------------------%
\newtheorem{definition}{Definition}   % 
\theoremstyle{definition}             % alter theorem style: <definition>
\newtheorem{program}{Program}         % [program]
\newtheorem{assumption}{Assumption}   % [assumption]
\newtheorem{example}{Example}         % [example]
\newtheorem{Algorithm}{Algorithm}     % [algorithm]
\newtheorem{policy}{Policy}           % [policy]
\newtheorem{problem}{Problem}         % [problem]
\theoremstyle{remark}                 % alter theorem style: <remark>
\newtheorem{remark}{Remark}           % [remark]
\theoremstyle{plain}                  % alter theorem style: <plain>
\newtheorem{theorem}{Theorem}         % [theorem]
\newtheorem{lemma}{Lemma}             % [lemma]
\newtheorem{corollary}{Corollary}     % [corollary]
%---------------------------------------------------------------%
\newcommand{\eq}{=}
\newcommand{\domZ}{\mathbb{Z}_{*}}
\newcommand{\domP}{\mathbb{Z}_{*}}
\newcommand{\vecOne}{\mathbf{1}}
\newcommand{\ind}{\mathbf{I}}
\newcommand{\mat}{\mathbf}
\newcommand{\Poisson}{\text{Poisson}}
\newcommand{\Bernoulli}{\text{Bernoulli}}
\newcommand{\define}{\triangleq}
\newcommand{\leadto}{\Rightarrow}
\newcommand{\vecG}{\boldsymbol}
\renewcommand{\vec}{\mathbf}
\DeclarePairedDelimiter{\set}{\{}{\}}
\DeclarePairedDelimiter{\norm}{|}{|}
\DeclarePairedDelimiter{\Inorm}{\|}{\|_1}
\DeclarePairedDelimiter{\Paren}{\bigg(}{\bigg)}
\DeclarePairedDelimiter{\Bracket}{\bigg[}{\bigg]}
\DeclarePairedDelimiter{\Brace}{\bigg\{}{\bigg\}}
%
\newcommand{\spaceblank}{\vskip 4mm}
\renewcommand{\baselinestretch}{1.4}
%---------------------------------------------------------------%
\newcommand{\AP}{\dagger}
\newcommand{\ES}{\ddagger}
\newcommand{\apSet}{\mathcal{K}}
\newcommand{\esSet}{\mathcal{M}}
\newcommand{\ccSet}{\mathcal{X}}
\newcommand{\jSpace}{\mathcal{J}}
\newcommand{\Stat}{\mathbf{S}}
\newcommand{\Obsv}{\mathcal{Y}}
\newcommand{\Policy}{\vecG{\Omega}}
\newcommand{\Delay}{\vecG{\mathcal{D}}}
\newcommand{\Baseline}{\vecG{\Pi}}
\newcommand{\algname}{\texttt{DecMDP}}

\newcommand{\comments}[1]{\spaceblank\noindent{\leavevmode\color{black}\em#1}}
\newcommand{\response}[1]{\spaceblank{\leavevmode\color{blue}#1}}
\newcommand{\thankyou}{Thank you for the comment}
\newcommand{\delete}[2]{}
\newcommand{\needref}[1]{{\leavevmode\color{red}[#1]}}
%
\newcommand{\wangr}[1]{{\leavevmode\color{orange}#1}}
\newcommand{\hongyc}[1]{{\leavevmode\color{purple}#1}}
\newcommand{\hongycCHK}[1]{{\leavevmode\color{black}#1}}
\newcommand{\tann}[1]{{\leavevmode\color{red}#1}}
\newcommand{\tannCHK}[1]{{\leavevmode\color{black}#1}}
%---------------------------------------------------------------%
\newcommand{\brlatency}{signaling latency}

\begin{document}
    \title{Reply to the Editor and Reviewers' Comments on Manuscript ID IoT-16294-2021}
    \author{}
    \maketitle

    %---------------------------------------------------------------%
    We thank the editor and reviewers for all the constructive comments. They have helped to improve the technical accuracy and presentation of the manuscript.
    In the revised manuscript, the main changes are emphasized in {\color{blue}blue} for the reading convenience.

    In this reply file, we first state the comments in {\em\color{black} italic and black}, and then respond to them in {\color{blue}blue}.
    Unless mentioned otherwise, the equation, figure and citation numbers refer to those in this reply file.
    %---------------------------------------------------------------%

    %-------------------------------- Response to Reviewer 1 -------------------------------%
    \section{Response to Reviewer 1}
    \comments{
        This paper investigates the distributed job dispatching in edge computing systems with complete technical analysis. The randomness in transmission latency is considered.
        I have the following suggestions and questions.
    }
    \response{
        We thank the reviewer 1 for the constructive and positive comments.
        We now address the comments in details below.
    }

    \comments{
        \textbf{Comment 1:} In the related, it is stated "staleness and failed transmission of system state information at the dispatcher of edge computing systems should be considered". How do you deal with the failed transmission of system state information?
    }
    \response{
        \textbf{Response R1-1:} \thankyou.
        In the original manuscript, the "failed transmission" occurred when "some broadcast information may be discarded by the routers after a certain number of hops" (Section III.B, 2nd paragraph), which results in partially observable state information for each AP (named as OSI in the Definition 5 of the manuscript).
        % We shall introduce a simple retransmission schema.
        Specifically, each AP makes dispatching decisions according to the state information of APs in its \emph{conflict AP set} and edge servers in its \emph{candidate server set}.
        The state information of other nodes is not requested, and the latency of state information transmission from those nodes to the AP is tolerable.
        %
        Moreover, the transmission failure of OSI is also tolerable in the proposed algorithm, which can actually be transferred to transmission latency.
        For example, if the $y$-th AP is in the \emph{conflict AP set} of the $k$-th AP and the state information transmission from the $y$-th AP failed, the $k$-th AP may request retransmissions until a successful reception.
        This will raise the transmission latency of OSI, and the random transmission latency of OSI is considered in this work.
        % \delete{v1}{
        %     However, our proposed solution framework could be easily extended to handle more general transmission failure situations with a simple retransmission schema.
        %     In one broadcast interval, if any AP could not receive the broadcast within a specific time limit, which is denoted as $\tau_0$ (random variable with support $\set{0, \dots, \hat{\tau}_0}$), it will request a retransmission of the broadcast from the corresponding AP or edge server.
        %     The AP or edge server will immediately repeat the broadcast when receive the retransmission request, and we denote the Round-Trip Time (RTT) as $\tau_1$ (random variable with support $\set{0, \dots,\hat{\tau}_1}$).
        %     As the retransmission request or the repeated broadcast may fail again, we denote retransmission times until success is denoted as $\theta_\tau$, and $\theta_\tau$ as the maximum retransmission times after which the transmission is believed to be received.
        %     Therefore, the broadcast interval $T$ must be set to bound the maximum time elapsed $\hat{\theta}_{\tau}(\hat{\tau}_0+\hat{\tau}_1)$ to guarantee the complete reception of OSI in one broadcast interval.
        %     Moreover, $(\hat{\tau}_0+\hat{\tau}_1)$ is limited because of the constraint of hops, and $\hat{\theta}_\tau$ is usually small.
        %     As a remark notice that the elapsed time $\theta_{\tau}(\tau_0+\tau_1)$ for the $k$-th AP at the $t$-th broadcast interval is exactly the \brlatency~$\Delay_k(t)$.
        %     Hence, our framework could handle the general packet transmission failure issue with the appended simple retransmission schema.
        % }
        We have appended a remark in Section III to address this issue in the revised manuscript.
    }

    \comments{
        \textbf{Comment 2:} Does edge servers need to broadcast their state information? In my understanding, only APs need to exchange state information as an edge server is collocated with an AP.
    }
    \response{
        \textbf{Response R1-2:} \thankyou.
        In face, the edge servers also broadcast their state information to the APs in the \emph{potential AP set}.
        It is stated in Section III.B in the original manuscript, "at the beginning of each broadcast interval, the local state information (LSI) of APs and edge servers are broadcast."
        It is necessary for the APs to collect task queueing information from edge servers, even it is outdated.
        % Intuitively, the job dispatcher on edge servers needs queue status information from the candidate servers to make job dispatching decisions.
    }

    \comments{
        \textbf{Comment 3:} Does job response include the job processing time?
    }
    \response{
        \textbf{Response R1-3:} \thankyou.
        Yes, the job response time includes the job processing time on the edge server, plus the uploading time from the AP to the edge server and the waiting time in the processing queue on the edge server.
        We have revised the manuscript to avoid potential misunderstanding.
    }

    \comments{
        \textbf{Comment 4:} Appearance of words in figures is in different sizes. Please revise them to be clear.
    }
    \response{
        \textbf{Response R1-4:} \thankyou.
        We have adjusted the font of the words appearing in the figures to the same size.
    }

    \comments{
        \textbf{Comment 5:} Please checked the presentation carefully. For example, it should be
        "staleness and failed transmission of system state information at the dispatcher of edge computing systems should be considered" but not "staleness and failed transmission of system state information at the dispatcher of a edge computing systems should be considered".
    }
    \response{
        \textbf{Response R1-5:} \thankyou.
        We have carefully checked the presentation used in the original manuscript and made the corresponding modifications in the revised version.
    }
    %-------------------------------- Response to Reviewer 1 -------------------------------%


    %-------------------------------- Response to Reviewer 2 -------------------------------%
    \section{Response to Reviewer 2}
    \comments{
        Since highly dynamic transmission latency and job arrival distribution make it difficult for Access Points (APs) to dispatch jobs to edge servers, this paper constructs a distributed job dispatching framework, where the APs cooperatively upload different types of jobs to different edge servers with random uploading latency.
        Considering the APs receive the partial system state information suffering from random signaling latency, a Partially Observable Markov Decision Process (POMDP) problem is formulated, subjected to the partially available and outdated global system state at the APs.
        In order to solve the constructed Bellman’s equations, the authors first propose a baseline policy to approximate the optimal value function.
        Then, the APs are partitioned into non-conflict subsets to realize distributed action update via alternative policy iteration.
        Sufficient analytical derivation on performance lower bound is a strong aspect of this work.
        However, the following issues should be clarified before publication.
    }

    \response{
        We thank the reviewer 2 for the constructive and positive comments.
        We now address the comments in details below.
    }

    \comments{
        \textbf{Comment 1:} The authors leveraged partially observable Markov decision process for problem formulation. According to Definition 5, “the observable state information (OSI) of the k-th AP is defined as the aggregation of LSIs of the APs in conflict AP set and the edge servers in candidate server set of the k-th AP”. However, the state information of the APs in non-conflict AP set and the edge servers not in candidate server set of the k-th AP is useless to the k-th AP. In other words, the k-th AP can observe all information related to it during each broadcast interval. To some extent, it can also be said that the AP has observed “global information”. What do the authors think of the global information for a single AP?
    }
    \response{
        \textbf{Response R2-1:} \thankyou.
        \hongyc{
            1. GSI/OSI, Direct and Indirect impact on job scheduling;
            2.
        }
        Actually, "the state information in non-conflict AP set and the edge servers not in candidate server set is useless" only holds in Problem P2, i.e., the problem formulated for each AP after \emph{AP partition} is established.
        Before \emph{AP partition}, each AP targets to solve problem P1 via equation (\ref{eqn:sp_0})
        {\small\begin{align}
            &V\Paren{\Stat(t)} =g\Paren{\Stat(t)}
                + \gamma\mathbb{E}_{\Delay}\bigg\{
                    \min_{\Policy(\Stat(t),\Delay(t))}
                    % \nonumber\\
                    \sum_{\Stat(t+1)} \Pr \Big\{ 
                        \Stat(t+1) \Big| \Stat(t), \Policy(\Stat(t), \Delay(t)) \Big\} \cdot V\Big(\Stat(t+1)\Big)
                    \bigg\},
            \label{eqn:sp_0}
        \end{align}}
        where the global state information (GSI) $\Stat(t)$ is engaged.
        With the GSI available on each AP, the update of dispatching actions could be simultaneously carried out on each AP by solving equation (\ref{eqn:sp_0}) once they observe the GSI, and the order of action update does not matter.
        However, this action update schema is not achievable as GSI is not available on each AP.
        The APs have to follow an \emph{alternative action update} schema with the help of \emph{AP partition}, where only a subset of APs can update dispatching actions in one broadcast interval by solving equation (\ref{eqn:sp_1})
        {\small
        \begin{align}
            \min_{ \tilde{\mathcal{A}}(t) }
            &\sum_{\Stat(t+1)} \Pr\Brace{
                \Stat(t+1) \Big| \Stat(t), \hat{\mathcal{A}}(t), \tilde{\mathcal{A}}(t)
            } \cdot W_{\Baseline}\Paren{\Stat(t+1)},
            \label{eqn:sp_1}
        \end{align}
        }
        while the others maintain the same actions in the previous broadcast interval.
        Hence, the observed information on each AP would still be considered partial information as OSI but not GSI.
        % Moreover, the schema which leverages all the broadcast information (i.e., global information), suffers from 1) not scalable; 2) broadcast interval would be larger; 3) randomness would increase regarding the transmission failure.
    }

    \comments{
        \textbf{Comment 2:} Does the proposed job dispatching framework consider the possibility of packet drop? For example, in Fig. 2, What if the information broadcast by the 2-nd AP is not received by the 1-st AP?
    }
    \response{
        \textbf{Response R2-2:} \thankyou.
        % The job dispatching framework could support possibility of packet drop, a.k.a. failed transmission of system state information.
        As explained in \textbf{Response R1-1}, if packet drip occurs in the transmission of OSI, retransmissions can be requested by the receiving AP.
        Hence, the packet drop can be transferred to receiving latency of OSI, which has been addressed by the proposed algorithm.
        % with a simple retransmission schema, the $1$-st AP will request a retransmission of the broadcast information from the $2$-nd AP until success.
        We have appended a remark in Section III to address this issue in the revised manuscript.
    }

    \comments{
        \textbf{Comment 3:} In Section VI, the authors mentioned "the distributions of job arrival, uploading latency and computation time, which are usually unknown in practice". However, the proposed reinforcement learning approach can estimate these values in reality. What are the challenges of Algorithm 2 not being able to estimate these values?
    }
    \response{
        \textbf{Response R2-3:} \thankyou.
        We are sorry for the misunderstanding.
        Because the network may not be stationary for a long time, it is difficult to predict the distribution of job arrival, uploading latency and computation time offline in advance.
        Our proposed algorithm provide an efficient method to track these statistics in an online manner.
        We have revised the sentence to avoid misunderstanding.
        % With the help of the proposed reinforcement learning algorithm, the unknown statistics (i.e., the distributions of job arrival, uploading latency and computation time) could be learned efficiently, and then the calculation of approximate value function $W_{\Policy}(\Stat)$ is feasible.
        % We have replaced "in practice" with a more precise expression "in advance" in the revised version.
    }

    %NOTE: normal issue
    \comments{
        \textbf{Comment 4:} The proposed MDP policy is compared with three kinds of baselines. Are there any state-of-the-art methods proposed recently?
    }
    \response{
        \textbf{Response R2-4:} \thankyou.
        As mentioned in the introduction section of the original manuscript, MDP has already been applied in scheduling algorithm design for edge computing systems \needref{Edge-MDP}.
        However, to the best of our knowledge, there are no existing methods proposed to solve job dispatching with outdated and partial information of the edge computing system.
        \hongyc{
            Because it is infeasible to solve the Bellman's equation in the considered scenario.
            In fact, the three baselines are commonly considered in the existing literature.
            For example, 
            % The three baselines used in the original manuscript are to profile the performance of the proposed solution framework \algname~under different parameter settings.
            % Specifically,
            % There is no exisiting state-of-the-art baseline, no such scheduling consider distributed scheduling with partial and stale information.
            % in the sensitivity study section
        }
    }

    \comments{
        \textbf{Comment 5:} It is recommended to enlarge the legends of Figures 5 and 7.
    }
    \response{
        \textbf{Response R2-5:} \thankyou.
        We have enlarged the legends of Figures 5 and 7.
    }
    %-------------------------------- Response to Reviewer 2 -------------------------------%


    \bibliographystyle{IEEEtran}
    % \bibliography{references.bib}
\end{document}